\documentclass[a4paper]{article}
\usepackage[utf8]{inputenc}
\usepackage{graphicx}
\usepackage{fontspec}

% Define your paper size
\def\realpaperwidth{4in}
\def\realpaperheight{2.5in}

% Define your paper size after cutting
\def\cutpaperwidth{3.3in}
\def\cutpaperheight{1.8in}

% Define your background image, which should include the overlap
\def\backgroundwidth{3.5in}
\def\backgroundheight{2.0in}
% Attention: If the height doesn't fit to the width, it will cause distortion.

% Define the area, in which the name text should appear
\def\textboxwidth{2.5in}
\def\textboxheight{2in}
% Want to adjust the vertical text position, do it here
\def\vnameoffset{0pt}

% Font type and color
\setmainfont{Playfair Display}
\def\selectedfontsize{Huge}
\usepackage{color}
\definecolor{Fontcolor}{rgb}{0.572,0.525,0.396} % r,g,b values from 0 to 1

% Change nothing here
\usepackage[%
paperwidth=\cutpaperwidth,%
paperheight=\cutpaperheight,%
margin=0in
]{geometry}

% Adjust here, which driver you use. Currently LuaLaTeX is chosen. Most people use pdflatex, but I recommend LuaLaTeX for this, because of the file encoding and font choice. Just uncomment the one, you use and comment out lualatex.
\usepackage[%
width=\realpaperwidth,
height=\realpaperheight,
landscape,
cam,
axes,
noinfo,
%dvips,
%pdftex,
%pdflatex,
%luatex,
lualatex,
%vtex,
%nodriver
center]{crop}

% Fix the page size and everything
\textheight=\cutpaperheight
\textwidth=\cutpaperwidth
\leftmargin=0pt


%%%%%%%%%%%%
%You're done
%%%%%%%%%%%%


% Simple fix for the indent at every new paragraph, which we don't need here.
\setlength{\parindent}{0pt}


%Personal chopline command, stolen from https://de.wikibooks.org/wiki/LaTeX-W%C3%B6rterbuch:_Serienbrief_mit_externen_Daten
\def\chopline#1;#2;#3 \\{
\def\nametitle{#1}
\def\prename{#2}
\def\surname{#3}
}
 
\newif\ifmore \moretrue
 

%\inputencoding{latin9} % not needed, if you use LuaLaTeX
\newread\quelle
\openin\quelle=addresses.csv %Here we get the addresses from
%\inputencoding{utf8} % not needed, if you use LuaLaTeX


\begin{document}
\pagenumbering{gobble}% no page numbering
\loop
\read\quelle to\zeile
\ifeof\quelle
\global\morefalse
\else{%
\expandafter\chopline\zeile\\%
\newpage% automatically ignored for first page
\hfill\hspace*{-0.75in}% prevent overfull boxes
\parbox[c][\cutpaperheight][c]{\cutpaperwidth}{% a box for the whole page content, to fit it more easily
\vskip -0.5in plus 1fill minus 1fill\relax% prevent overfull and center
\hfill\hspace*{-0.5in}% prevent overfull and center
\mbox{% prevent line break
\parbox[c][\backgroundheight][c]{\backgroundwidth}{% box to center the background image
\mbox{}\hspace*{-0.5in}\hfill% prevent overfull and center
%
%%%%%%%%%%%%%%%%%%%%%%%%%%%%%%%%%%%%%%%%%%%%%%%%%%%%%
%
% Here you might need to adjust the parameters of includegraphics
% Note also the \textit command, that switches to italic text. You can change it.
%
\includegraphics[%
width=\backgroundwidth,%
height=\backgroundheight,%
page=2,%
]{background.pdf}%
%
%%%%%%%%%%%%%%%%%%%%%%%%%%%%%%%%%%%%%%%%%%%%%%%%%%%%%
%
\hfill\hspace*{-0.5in}\mbox{}% prevent overfull and center
}% parbox background image
\hspace*{-\backgroundwidth}% make text appear on top of background
\parbox[c][\backgroundheight][c]{\backgroundwidth}{% helper box to fit text on background
\mbox{}\vskip -2.4em\relax\mbox{}% compensate for empty lines caused by mbox, which is needed to suppress skip suppression = force skips to appear
\vskip \dimexpr\vnameoffset\relax\mbox{}% adjust the text position vertically
\begin{center}%
\parbox[c][\textboxheight][c]{\textboxwidth}{% actual text box
\begin{center}%
%
%%%%%%%%%%%%%%%%%%%%%%%%%%%%%%%%%%%%%%%%%%%%%%%%%%%%%
%
% The actual text is here. As you see, there is a linebreak. If you like it, keep it, else delete it.
%
\begin{\selectedfontsize}\textit{\color{Fontcolor}\vspace{0.3em}% Make a bit more vertical line space, depending on the fontsize
\nametitle\ \prename\\\surname}% Here we insert the data from the address list. See chopline command
\end{\selectedfontsize}%
%
%%%%%%%%%%%%%%%%%%%%%%%%%%%%%%%%%%%%%%%%%%%%%%%%%%%%%
%
\end{center}%
}% parbox textbox
\end{center}%
\hskip -0.45\dimexpr\backgroundwidth plus 1fill\relax%
\mbox{}\unskip%
}% parbox helper box
}% mbox no line break
\hspace*{-0.5in}%
\hfill%
\mbox{}%
\vskip -0.5in plus 1fill minus 1fill\relax% prevent overfull and center
}% parbox page content
\hspace*{-0.75in}\hfill\mbox{}% prevent overfull and center
}%else
\vskip 0pt% create a paragraph break
\fi%
\ifmore\repeat%
\closein\quelle%
\end{document}
